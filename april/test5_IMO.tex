\documentclass{article}

\usepackage{mathtools,amssymb}
\usepackage{enumitem}
% \usepackage{fullpage}
\usepackage{fancyvrb}
\usepackage{hyperref}


\begin{document}
\thispagestyle{empty}

\begin{center}
  \textbf{\Large Test 5: IMO}
  \\ \vspace{1em}
  \textbf{\large April Camp 2023}
  \\ \vspace{1em}
  \textbf{\large Time: $4\frac{1}{2}$ hours}
\end{center}

\bigskip
\vfill

\begin{enumerate}[itemsep=\medskipamount]

\item % IMOSL N2 2022
Find all positive integers $n$ such that
\[ n! \ \mathrel{\biggl\vert} \prod_{\substack{p < q \leq n \\ p,q\ \text{primes}}} (p+q). \]~


\item % IMOSL A4 2022
Let $n \geq 3$ be an integer, and let $x_1, x_2, \dotsc, x_n$ be real numbers in the interval $[0,1]$.
Let $s = x_1+x_2+\dotsb+x_n$, and assume that $s \geq 3$.
Prove that there exist integers $i$ and $j$ with $1 \leq i < j \leq n$ such that
\[ 2^{j-i} x_i x_j > 2^{s-3}. \]~


\item % IMOSL C6 2022
Let $n$ be a positive integer.
We start with $n$ piles of pebbles, each initially containing a single pebble.
One can perform moves of the following form: choose two piles, take and equal number of pebbles from each pile and form a new pile out of these pebbles.
For each positive integer $n$, find the smallest number of non-empty piles that one can obtain by performing a finite sequence of moves of this form.


\end{enumerate}


\vfill
% ASCII art
\centering
\fontsize{5pt}{4.5pt}
\begin{BVerbatim}
                               db
                              dbdb
                             db  db
                            dbdbdbdb
                           db      db
                          dbdb    dbdb
                         db  db  db  db
                        dbdbdbdbdbdbdbdb
                       db              db
                      dbdb            dbdb
                     db  db          db  db
                    dbdbdbdb        dbdbdbdb
                   db      db      db      db
                  dbdb    dbdb    dbdb    dbdb
                 db  db  db  db  db  db  db  db
                dbdbdbdbdbdbdbdbdbdbdbdbdbdbdbdb
               db                              db
              dbdb                            dbdb
             db  db                          db  db
            dbdbdbdb                        dbdbdbdb
           db      db                      db      db
          dbdb    dbdb                    dbdb    dbdb
         db  db  db  db                  db  db  db  db
        dbdbdbdbdbdbdbdb                dbdbdbdbdbdbdbdb
       db              db              db              db
      dbdb            dbdb            dbdb            dbdb
     db  db          db  db          db  db          db  db
    dbdbdbdb        dbdbdbdb        dbdbdbdb        dbdbdbdb
   db      db      db      db      db      db      db      db
  dbdb    dbdb    dbdb    dbdb    dbdb    dbdb    dbdb    dbdb
 db  db  db  db  db  db  db  db  db  db  db  db  db  db  db  db
dbdbdbdbdbdbdbdbdbdbdbdbdbdbdbdbdbdbdbdbdbdbdbdbdbdbdbdbdbdbdbdb
\end{BVerbatim}

\end{document}

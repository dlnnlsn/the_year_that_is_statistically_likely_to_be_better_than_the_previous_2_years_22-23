\documentclass{article}

\usepackage{mathtools,amsfonts}
\usepackage{enumitem}
% \usepackage{fullpage}
\usepackage{fancyvrb}
\usepackage{hyperref}


\newcommand{\parens}[1]{\left(#1\right)}

\begin{document}
\thispagestyle{empty}

\begin{center}
  \textbf{\Large Test 4}
  \\ \vspace{1em}
  \textbf{\large April Camp 2023}
  \\ \vspace{1em}
  \textbf{\large Time: $4\frac{1}{2}$ hours}
\end{center}

% \bigskip
\vfill

\begin{enumerate}[itemsep=\fill]

\item % EMC 2014
Prove that there are infinitely many positive integers which can't be expressed as $a^{d(a)}+b^{d(b)}$ where $a$ and $b$ are positive integers and $d(n)$ denotes the number of positive divisors of a positive integer $n$.


\item % ThMO 2014 Q5
Determine the largest real number $k$ such that the inequality
\[ \parens{k+\frac{a}{b}} \parens{k+\frac{b}{c}} \parens{k+\frac{c}{a}} \leq \parens{\frac{a}{b}+\frac{b}{c}+\frac{c}{a}} \parens{\frac{b}{a}+\frac{c}{b}+\frac{a}{c}} \]
holds for all positive real numbers $a$, $b$, and $c$.


\item % IMOSL C3 2022
In each square of a garden shaped like a $2022 \times 2022$ board, there is initially a tree of height $0$.
A gardener and a lumberjack take turns playing the following game, with the gardener taking the first turn:
\begin{itemize}
  \item The gardener chooses a square in the garden. Each tree on that square and all the surrounding squares (of which there are at most eight) then becomes one unit taller.
  \item the lumberjack then chooses four different squares on the board. Each tree of positive height on those squares then becomes one unit shorter.
\end{itemize}
We call a tree \emph{majestic} if its height is at least $10^6$.
Determine the largest number $K$ such that the gardener can ensure that there are eventually $K$ majestic trees in the garden, no matter how the lumberjack plays.


\item % IMOSL G4 2022
Let $ABC$ be an acute-angled triangle with $AC > BC$, let $O$ be its circumcentre, and let $D$ be a point on the segment $BC$.
The line through $D$ perpendicular to $BC$ intersects the lines $AO$, $AC$, and $AB$ at $W$, $X$, and $Y$ respectively.
The circumcircles of $AXY$ and $ABC$ intersect again at $Z \neq A$.
Prove that if $OW = OD$, then $DZ$ is tangent to the circle $AXY$.


\end{enumerate}


\vfill
% ASCII art
\centering
\fontsize{3pt}{2.75pt}
\begin{BVerbatim}
                  ____                    
             ____ \__ \
             \__ \__/ / __
             __/ ____ \ \ \    ____
            / __ \__ \ \/ / __ \__ \
       ____ \ \ \__/ / __ \/ / __/ / __
  ____ \__ \ \/ ____ \/ / __/ / __ \ \ \
  \__ \__/ / __ \__ \__/ / __ \ \ \ \/
  __/ ____ \ \ \__/ ____ \ \ \ \/ / __
 / __ \__ \ \/ ____ \__ \ \/ / __ \/ /
 \ \ \__/ / __ \__ \__/ / __ \ \ \__/
  \/ ____ \/ / __/ ____ \ \ \ \/ ____
     \__ \__/ / __ \__ \ \/ / __ \__ \
     __/ ____ \ \ \__/ / __ \/ / __/ / __
    / __ \__ \ \/ ____ \/ / __/ / __ \/ /
    \/ / __/ / __ \__ \__/ / __ \/ / __/
    __/ / __ \ \ \__/ ____ \ \ \__/ / __
   / __ \ \ \ \/ ____ \__ \ \/ ____ \/ /
   \ \ \ \/ / __ \__ \__/ / __ \__ \__/
    \/ / __ \/ / __/ ____ \ \ \__/
       \ \ \__/ / __ \__ \ \/
        \/      \ \ \__/ / __
                 \/ ____ \/ /
                    \__ \__/
                    __/
\end{BVerbatim}

\end{document}

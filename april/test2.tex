\documentclass{article}

\usepackage{mathtools,amsfonts}
\usepackage{enumitem}
% \usepackage{fullpage}
\usepackage{fancyvrb}
\usepackage{hyperref}


\begin{document}
\thispagestyle{empty}

\begin{center}
  \textbf{\Large Test 2}
  \\ \vspace{1em}
  \textbf{\large April Camp 2023}
  \\ \vspace{1em}
  \textbf{\large Time: $4\frac{1}{2}$ hours}
\end{center}

\bigskip
\vfill

\begin{enumerate}[itemsep=2\bigskipamount]

\item % Belarus 2002 Final Round Category D
There is a set $M$ of $20$ distinct real numbers.
It is known that for any two numbers $a,b \in M$ with $a < b$ there exists a number $x \in M$ such that $a < -x < b$.
How many positive numbers can be in $M$?


\item % PAMOSL N1 2022
Find all positive integers $a$, $b$, and $p$, where $p$ is prime, such that
\[ a^4 +b^4 +a^2b^2 = p^3. \]



\item % IMOSL G2 2022
In the acute-angled triangle $ABC$, the point $F$ is the foot of the altitude from $A$, and $P$ is a point on the segment $AF$.
The lines through $P$ parallel to $AC$ and $AB$ meet $BC$ at $D$ and $E$ respectively.
Points $X \neq A$ and $Y \neq A$ lie on circles $ABD$ and $ACE$ respectively such that $DA = DX$ and $EA = EY$.
Prove that $B$, $C$, $X$, and $Y$ are concyclic.


\item % IMOSL C5 2022
Let $m$ and $n$ be integers greater than $1$, let $X$ be a set with $n$ elements, and let $X_1, \dotsc, X_m$ be pairwise distinct non-empty subsets of $X$.
A function $f : X \to \{1, 2, \dotsc, n+1\}$ is called \emph{nice} if there exists an index $k$ such that
\[ \sum_{x \in X_k} f(x) > \sum_{x \in X_i} f(x) \quad\text{for all}\ i \neq k. \]
Prove that the number of nice functions is at least $n^n$.


\end{enumerate}


\bigskip
\vfill
% ASCII art
\centering
\fontsize{3pt}{2.75pt}
\begin{BVerbatim}
                        __/\__ 
                        \    /   
                  __/\__/    \__/\__
                  \                /
                  /_              _\
                    \            /
      __/\__      __/            \__      __/\__
      \    /      \                /      \    /
__/\__/    \__/\__/                \__/\__/    \__/\__
\                                                    /
/_                                                  _\
  \                                                /
__/                                                \__ 
\                                                    /
/_  __                                          __  _\
  \/  \                                        /  \/
      /_                                      _\
        \                                    /
      __/                                    \__
      \                                        /
__/\__/                                        \__/\__
\                                                    /
/_                                                  _\
  \                                                /
__/                                                \__
\                                                    /
/_  __      __  __                  __  __      __  _\
  \/  \    /  \/  \                /  \/  \    /  \/
      /_  _\      /_              _\      /_  _\
        \/          \            /          \/
                  __/            \__
                  \                /
                  /_  __      __  _\
                    \/  \    /  \/
                        /_  _\
                          \/
\end{BVerbatim}

\end{document}

\documentclass{article}

\usepackage{mathtools,amsfonts}
\usepackage{enumitem}
\usepackage[cm]{fullpage}
\usepackage{fancyvrb}
\usepackage{hyperref}

\newcommand{\abs}[1]{\left|#1\right|}

\begin{document}
\thispagestyle{empty}

\begin{center}
  \textbf{\Large January Advanced Monthly Problem Set}
  \\ \vspace{1em}
  \textbf{\large Due: 20 January 2023}
\end{center}

\bigskip

\begin{enumerate}[itemsep=\fill]

 
\item % RuMO 2015/16 G9 Q6
We say that a non-empty set $A$ consisting of positive integers is \textit{complete} if for any positive integers $a$ and $b$ such that $a+b\in A$, the number $ab$ also lies in $A$ (the numbers $a$ and $b$ are not required to be distinct or to belong to $A$). Find all complete sets.


\item % RuMO 2015/16 G9 Q4
Dylan has 11 metal pieces indistinguishable in appearance.
He knows that their weights in some order are $1,2,\dotsc,11$ kg.
He also has a bag that breaks if it contains more than 11 kg (it will not break if it contains exactly 11 kg).
Liam knows the weight of each piece and he wants to convince Dylan beyond a shadow of a doubt about which piece has the 1 kg weight.
By a `move', Liam can put several pieces into Dylan's bag to demonstrate that the bag is still not broken (he may not break the bag!).
Find the least number of moves Liam needs to convince Dylan.


\item % Liam
We say that an infinite sequence $a_1, a_2, \dotsc$ is a \emph{binary sequence} if each $a_i$ is equal to $1$ or to $-1$.
Find all binary sequences such that \[ \max\{\abs{a_1}, \abs{2a_1+a_2}, \abs{3a_1+2a_2+a_3}, \abs{4a_1+3a_2+2a_3+a_4}, \dotsc\} = 1. \]


\item % LB-2017-3
Let $n \geq 5$ be an odd positive integer.
Let $P$ be a point in the sameplace as the convex $n$-gon $A_1 A_2 \dotsm A_n$, satisfying the following conditions:
\begin{itemize}
	\item $\angle PA_1A_2 = \angle PA_2A_3 = \dotsb = \angle PA_{n-1}A_n = \angle PA_nA_1$
  \item $\angle PA_1A_3 =\angle PA_2A_4 = \dotsb = \angle PA_{n-1}A_1 = \angle PA_nA_2$
  \item for each $1\leq i\leq n$, $P$ and $A_i$ are on opposite sides of the line $A_{i-1}A_{i+1}$.
\end{itemize}
Show that $PA_1 = PA_2 = \dotsb = PA_n$ and $A_1A_2 = A_2A_3 = \dotsb = A_{n-1}A_n = A_nA_1$.


\item % JM-2016-3
Let $n \geq 3$ be an integer and let $x_1, x_2, \dotsc, x_n$ be positive real numbers such that
\[ x_1 + x_2 + \dotsb + x_n = \frac{1}{x_1^2} + \frac{1}{x_2^2} +\dotsc + \frac{1}{x_n^2} \quad \text{and} \quad x_1^2 + x_2^2 + \dotsb + x_n^2 = \frac{1}{x_1} + \frac{1}{x_2} +\dotsc + \frac{1}{x_n}. \]
Prove that $x_1 = x_2 = \dotsb = x_n =1$.


\item % 


\item % 


\item % 

\end{enumerate}


\vfill
\small
\begin{itemize}
	\item Submit your solutions at \href{https://forms.gle/9EBuvypU7ppDmprt8}{https://forms.gle/9EBuvypU7ppDmprt8}
	\item Submit each question in a single separate PDF file (with multiple pages if necessary).
	\item If you take photographs of your work, use a document scanner such as Office Lens to convert to PDF.
	\item If you have multiple PDF files for a question, combine them using software such as PDFsam.
\end{itemize}


% \vfill
% % ASCII art
% \centering
% \small
% \begin{BVerbatim}
% % INSERT ART HERE
% \end{BVerbatim}

\end{document}

\documentclass{article}

\usepackage{mathtools,amsfonts}
\usepackage{enumitem}
\usepackage{fullpage}
\usepackage{fancyvrb}
\usepackage{hyperref}


\begin{document}
\thispagestyle{empty}

\begin{center}
  \textbf{\Large January Intermediate Monthly Problem Set}
  \\ \vspace{1em}
  \textbf{\large Due: 13 January 2023}
\end{center}

\bigskip \bigskip

\begin{enumerate}[itemsep=\fill]

\item % 

 
\item % 


\item %


\item % 


\item % RuMO 2015/16 G9 Q6
We say that a non-empty set $A$ consisting of positive integers is \textit{complete} if for any positive integers $a$ and $b$ such that $a+b\in A$, the number $ab$ also lies in $A$ (the numbers $a$ and $b$ are not required to be distinct or to belong to $A$). Find all complete sets.

\item %

\end{enumerate}


\vfill
\small
\begin{itemize}
	\item Submit your solutions at \href{https://forms.gle/9EBuvypU7ppDmprt8}{https://forms.gle/9EBuvypU7ppDmprt8}
	\item Submit each question in a single separate PDF file (with multiple pages if necessary).
	\item If you take photographs of your work, use a document scanner such as Office Lens to convert to PDF.
	\item If you have multiple PDF files for a question, combine them using software such as PDFsam.
\end{itemize}

\vfill
% ASCII art
\centering
\small
\begin{BVerbatim}
% INSERT ART HERE
\end{BVerbatim}

\end{document}

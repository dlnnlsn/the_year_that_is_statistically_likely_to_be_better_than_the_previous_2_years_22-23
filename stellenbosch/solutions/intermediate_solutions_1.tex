\documentclass{article}

\usepackage{mathtools,amsfonts}
\usepackage{enumitem}
\usepackage{fullpage}
\usepackage{fancyvrb}
\usepackage{hyperref}


\begin{document}
\thispagestyle{empty}

\begin{center}
  \textbf{\Large Intermediate Test 1}
  \\ \vspace{1em}
  \textbf{\large Stellenbosch Camp 2022}
  \\ \vspace{1em}
  \textbf{\large Time: $2\frac{1}{2}$ hours}
\end{center}

\bigskip

\begin{enumerate}[itemsep=\fill]
	


\item % Tim, 2022
\textbf{Solution:} Note that the $1\times1$ square will have to be 'next' to an $\mathbf{L}$-shaped tile to form a $2\times2$ square. We can then consider the amount of ways to tile the $2\times5$ board with a $2\times2$ square and two $\mathbf{L}$-shaped tiles. Note that the square will have to be placed on the board such that the amount of tiles to the left and right of it are multiples of $3$, leaving two possible locations for the square on the edges of the board. The resulting $2\times3$ rectangle can be tiled in two ways with the remaining $\mathbf{L}$-shaped tiles. Considering the $4$ rotations of the $2\times2$ square as unique tilings, the $2$ locations the square can be, as well as the $2$ ways the resulting rectangle can be tiled, there are $16$ possible tilings. 


\item %

\item %


\item % 


\item % Liam, 2022
Where $p$ is a prime and $n\in\mathbb{N}$, find all solutions to the equation \[p^2 = 2^n + 1.\]

\textbf{Solution:} The only solution is $p = 3$ and $n = 3$.
\\ Clearly $p>2$. Then 
\begin{align*}
	p^2 &= 2^n + 1
	\\ \iff p^2-1 &= 2^n
	\\ \iff (p-1)(p+1) &= 2^n
\end{align*}
so that both $p-1$ and $p+1$ are powers of two. Since at least one of $p-1$ and $p+1$ is not divisible by $4$, it has to be equal to $2$.
\\ Now $p+1 = 2$ does not give a prime, and $p-1 = 2$ yields that $p =3$ and $n=3$ as the only solution. 


\end{enumerate}


% ASCII art
\centering
\small
\begin{BVerbatim}
% Insert art here
\end{BVerbatim}

\end{document}

\documentclass{article}

\usepackage{mathtools,amsfonts}
\usepackage{enumitem}
\usepackage{fullpage}
\usepackage{fancyvrb}
\usepackage{hyperref}


\begin{document}
\thispagestyle{empty}

\begin{center}
  \textbf{\Large Intermediate Test 1}
  \\ \vspace{1em}
  \textbf{\large Stellenbosch Camp 2022}
  \\ \vspace{1em}
  \textbf{\large Time: $2\frac{1}{2}$ hours}
\end{center}

\bigskip

\begin{enumerate}[itemsep=\fill]

\item % Tim, 2022
Note that the $1\times1$ square will have to be 'next' to an $\mathbf{L}$-shaped tile to form a $2\times2$ square. We can then consider the amount of ways to tile the $2\times5$ board with a $2\times2$ square and two $\mathbf{L}$-shaped tiles. Note that the square will have to be placed on the board such that the amount of tiles to the left and right of it are multiples of $3$, leaving two possible locations for the square on the edges of the board. The resulting $2\times3$ rectangle can be tiled in two ways with the remaining $\mathbf{L}$-shaped tiles. Considering the $4$ rotations of the $2\times2$ square as unique tilings, the $2$ locations the square can be, as well as the $2$ ways the resulting rectangle can be tiled, there are $16$ possible tilings. 


\item
\textbf{Solution:} 
\begin{enumerate}

\end{enumerate}



\item %


\item % 


\item %


\end{enumerate}


% ASCII art
\centering
\small
\begin{BVerbatim}
% Insert art here
\end{BVerbatim}

\end{document}

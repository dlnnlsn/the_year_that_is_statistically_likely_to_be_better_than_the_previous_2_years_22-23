\documentclass{article}

\usepackage{mathtools,amsfonts}
\usepackage{enumitem}
\usepackage{fullpage}
\usepackage{fancyvrb}
\usepackage{hyperref}
\usepackage{parskip}


\begin{document}
\thispagestyle{empty}

\begin{center}
  \textbf{\Large Intermediate Test 1 Solutions}
  \\ \vspace{1em}
  \textbf{\large Stellenbosch Camp 2022}
\end{center}

\bigskip

\begin{enumerate}[itemsep=24pt]

\item % Tim, 2022
\textbf{Solution:} Note that the tile in the same column as the $1\times1$ square will have to be covered by an $\mathbf{L}$-shaped tile to form a $2\times2$ square. We can then consider the amount of ways to tile the $2\times5$ board with a $2\times2$ square and two $\mathbf{L}$-shaped tiles. Note that the square will have to be placed on the board such that the amount of tiles to the left and right of it are multiples of $3$, leaving two possible locations for the square on the edges of the board. The resulting $2\times3$ rectangle can be tiled in two ways with the remaining $\mathbf{L}$-shaped tiles. Considering the $4$ rotations of the $2\times2$ square as unique tilings, the $2$ locations the square can be, as well as the $2$ ways the resulting rectangle can be tiled, there are $16$ possible tilings. 


\item % Tim, 2022
Let the altitude from point $A$ meet $BC$ at $D$, and let this length be $h$. Note that in $\triangle ADB$, $AB$ is the hypotenuse of a right angled triangle, meaning that $h = AD \leq AB = 1$. The area of $\triangle ABC$ is $\dfrac{1}{2}AD.BC = \dfrac{1}{2}h \leq \dfrac{1}{2}.1$. This is achieved in the degenerate case where $D = B$ and $\angle ABC = 90^{\circ}$.


\item % 
Given an acute angled triangle with sides of lengths $a,b$ and $c$. Prove that \[a^2 + b^2 > c^2.\]

\textbf{Solution:} Consider an acute-angled triangle $ABC$ such that the perpendicular from $A$ intersects $BC$ at $D$. Let $AB=a$, $BC=b$ and $CA=c$. Notice that $AB^{2}$ = $BD^{2} + DA^{2}$ by Pythagoras' Theorem.  and $BC^{2}$ = $(BD + DC)^{2}$ = $BD^{2} + 2BD\times DC + DC^{2}$. Thus
\[ AB^{2} + BC^{2} = DA^{2} + DC^{2} + 2BD \times DC + 2BD^{2}. \]
Moreover, since $CA^{2} = DA^{2} + DC^{2}$ by Pythagoras' Theorem. We have
\[ AB^{2} + BC^{2} = CA^{2} + 2BD \times DC + 2BD^{2} > CA^{2}. \]
Therefore $a^{2} + b^{2} = AB^{2} + BC^{2} > CA^{2} = c^{2}$.


\item % Liam, 2022
Where $p$ is a prime and $n\in\mathbb{N}$, find all solutions to the equation \[p^2 = 2^n + 1.\]

\textbf{Solution:} The only solution is $p = 3$ and $n = 3$.
\\ Clearly $p>2$. Then 
\begin{align*}
	p^2 &= 2^n + 1
	\\ \iff p^2-1 &= 2^n
	\\ \iff (p-1)(p+1) &= 2^n
\end{align*}
so that both $p-1$ and $p+1$ are powers of two. Since at least one of $p-1$ and $p+1$ is not divisible by $4$, it has to be equal to $2$.
\\ Now $p+1 = 2$ does not give a prime, and $p-1 = 2$ yields that $p =3$ and $n=3$ as the only solution. 


\item
There are at least 3 people at a party. All of them have an even number of friends, where friendship is mutual. Show that there are 3 of them who each have the same number of friends.

\textbf{Solution:} If there are $2n$ people, then they can have $0,2,...$ or $2n-2$ friends. There are thus $n$ options for how many friends each person can have. If there is someone with 0 friends, there cannot be 2 people with $2n-2$ friends and if there is someone with $2n-2$ friends there cannot be 2 people with 0 friends. We now consider 2 cases:
\begin{enumerate}
\item There is a number of friends that no one has (either 0 or $2n-2$). Then there are $n-1$ options for how many friends people have, and $2n$ people. Thus, there must be 3 people that have the same number of friends (by pigeonhole principle).
\item There is one person who has 0 friends and one person who has $2n-2$. Then there are $2n-2$ other people, and $n-2$ options for how many friends they have. Again, there must now be 3 people that have the same number of friends.
\end{enumerate}
If instead there are $2n+1$ people, they can have $0,2,...$ or $2n$ friends. There are thus $n+1$ options for how many friends each person can have. If someone has 0 friends, no one can have $2n$. Then we have $n$ options with $2n+1$ people, so there must be 3 people that have the same number of friends.

\end{enumerate}

\end{document}
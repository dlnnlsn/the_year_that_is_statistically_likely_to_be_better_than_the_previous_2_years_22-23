\documentclass{article}

\usepackage{mathtools,amsfonts}
\usepackage{enumitem}
\usepackage{fullpage}
\usepackage{fancyvrb}
\usepackage{hyperref}
\usepackage{pgfplots}
\pgfplotsset{compat=1.15}
\usepackage{mathrsfs}
\usetikzlibrary{arrows}

\begin{document}
\thispagestyle{empty}
\definecolor{wrwrwr}{rgb}{0.3803921568627451,0.3803921568627451,0.3803921568627451}

\begin{center}
  \textbf{\Large <LEVEL> Test <NUMBER>}
  % LEVEL is Senior, Intermediate or Beginner
  % NUMBER is the test number: 1, 2, etc.
  \\ \vspace{1em}
  \textbf{\large Stellenbosch Camp 2022}
  \\ \vspace{1em}
  \textbf{\large Time: $2\frac{1}{2}$ hours}
\end{center}

\bigskip

\begin{enumerate}[itemsep=\fill]

\item % Source of problem
Problem statement


\item %


\item % 


\item % 


\item %  IMO Shortlist 1998 G3
Let $I$ be the incenter of triangle $\triangle ABC$.
Let $D$, $E$ and $F$ be the points of tangency of the incircle of $\triangle ABC$ with $BC$, $CA$ and $AB$, respectively.
The line $\ell$ passes through $B$ and is parallel to $DF$.
The lines $ED$ and $EF$ intersect $\ell$ at the points $X$ and $Y$.
Prove that $\angle XIY$ is acute.

\textbf{Solution:} 



\begin{center}

\begin{tikzpicture}[line cap=round,line join=round,>=triangle 45,x=0.3cm,y=0.3cm]
\clip(-14,-18) rectangle (18,14);
\draw [line width=1pt,color=wrwrwr] (-7.48418072213211,11.943124790001734)-- (9.851072169964434,-4.109768681891099);
\draw [line width=1pt,color=wrwrwr] (9.851072169964434,-4.109768681891099)-- (-10.910046034501077,-4.1752906960620875);
\draw [line width=1pt,color=wrwrwr] (-10.910046034501077,-4.1752906960620875)-- (-7.48418072213211,11.943124790001734);
\draw [line width=1pt,color=wrwrwr] (-4.120771171316693,1.3403872874828022) circle (5.494223694538245);
\draw [line width=1pt,color=wrwrwr] (-0.3877344450702902,5.371631612269011)-- (-4.103431507507615,-4.153809045165356);
\draw [line width=1pt,color=wrwrwr,domain=-23.37794930246684:24.80881197642785] plot(\x,{(--109.10645878569908-9.525440657434366*\x)/-3.715697062437325});
\draw [line width=1pt,color=wrwrwr,domain=-23.37794930246684:24.80881197642785] plot(\x,{(--50.04075544833976--2.8889978704593577*\x)/9.107212667095684});
\draw [line width=1pt,color=wrwrwr,domain=-23.37794930246684:24.80881197642785] plot(\x,{(-49.627514715624955-6.636442786975009*\x)/5.391515604658359});
\draw [line width=1pt,color=wrwrwr] (-4.120771171316693,1.3403872874828022)-- (15.51776782456035,10.417188783846877);
\draw [line width=1pt,color=wrwrwr] (-4.120771171316693,1.3403872874828022)-- (5.312705659956814,-15.744178093192318);
\draw [line width=1pt,color=wrwrwr] (-4.120771171316693,1.3403872874828022)-- (9.851072169964434,-4.109768681891099);
\draw [line width=1pt,color=wrwrwr] (-4.120771171316693,1.3403872874828022)-- (-0.3877344450702902,5.371631612269011);
\draw [line width=1pt,color=wrwrwr] (-4.120771171316693,1.3403872874828022)-- (-4.103431507507615,-4.153809045165356);


\draw [fill=black] (-7.48418072213211,11.943124790001734) circle (1pt);
\draw[color=black, left] (-7.48418072213211,11.943124790001734) node {$A$};
\draw [fill=black] (9.851072169964434,-4.109768681891099) circle (1pt);
\draw[color=black,right] (9.851072169964434,-4.109768681891099) node {$B$};
\draw [fill=black] (-10.910046034501077,-4.1752906960620875) circle (1pt);
\draw[color=black,left] (-10.910046034501077,-4.1752906960620875) node {$C$};
\draw [fill=black] (-4.120771171316693,1.3403872874828022) circle (1pt);
\draw[color=black,left] (-4.120771171316693,1.3403872874828022) node {$I$};
\draw [fill=black] (-0.3877344450702902,5.371631612269011) circle (1pt);
\draw[color=black,above] (-0.3877344450702902,5.371631612269011) node {$F$};
\draw [fill=black] (-4.103431507507615,-4.153809045165356) circle (1pt);
\draw[color=black,below left] (-4.103431507507615,-4.153809045165356) node {$D$};
\draw [fill=black] (-9.494947112165974,2.4826337418096536) circle (1pt);
\draw[color=black,below left] (-9.494947112165974,2.4826337418096536) node {$E$};
\draw[color=black] (2.0,-20.0) node {$\ell$};
\draw [fill=black] (15.51776782456035,10.417188783846877) circle (1pt);
\draw[color=black,above left] (15.51776782456035,10.417188783846877) node {$Y$};
\draw [fill=black] (5.312705659956814,-15.744178093192318) circle (1pt);
\draw[color=black,right] (5.312705659956814,-15.744178093192318) node {$X$};
\end{tikzpicture}
\end{center}
Note that if  $\triangle XIY$ is acute then $\angle XIY$ is acute. So the problem can be reduced to proving the inequality
\begin{flalign} %Keep this environment I need the equation annotation
  XI^2 +IY^2 > XY^2 \label{eqn:e1}.
\end{flalign}
We notice that $IF=ID$ (radii) and $BF=BD$ (tangents), therefore $DIFB$ is a kite, therefore $IB\perp FD$, therefore $IB\perp XY$ (corresponding angles, $DF||XY$). We can now reduce our proof requirement (\ref{eqn:e1})

\begin{flalign}
  && XI^2 +IY^2 &> XY^2 &\nonumber\\
  &\Leftrightarrow& XB^2+BI^2+BI^2+BY^2 &> (XB+BY)^2 & (\text{Pythagoras})\nonumber\\
  &\Leftrightarrow& XB^2+BY^2+2BI^2 &> XB^2+2XB \cdot BY+ BY^2 &\nonumber\\
  &\Leftrightarrow& BI^2 &> XB \cdot BY. &\label{eqn:e2}
\end{flalign}

Notice that $\angle BXD = \angle FDE = \angle AFE = \angle BFY$ (corr. $\angle$s, tan-chord, vert-op respectively). Similarly we see $\angle BDX = \angle EDC = \angle EFD = \angle BYF$. Therefore $\triangle BDX ||| \triangle BYF$ (2 $\angle$s equal). This yields the equality of ratios

\begin{flalign*}
  &&\frac{XB}{BD} &= \frac{FB}{BY}&\\
  &\Leftrightarrow&XB \cdot BY &= FB \cdot BD = FB^2.&
\end{flalign*}
We deduce
\begin{flalign*}
  XB \cdot BY = FB^2 = BI^2 - IF^2 < BI^2.
\end{flalign*}
So (\ref{eqn:e2}) is indeed satisfied.




\end{enumerate}


% ASCII art
\centering
\small
\begin{BVerbatim}
% Insert art here
\end{BVerbatim}

\end{document}

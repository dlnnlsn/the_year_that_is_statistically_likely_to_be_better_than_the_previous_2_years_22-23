\documentclass{article}

\usepackage{mathtools,amsfonts}
\usepackage{enumitem}
\usepackage{fullpage}
\usepackage{fancyvrb}
\usepackage{hyperref}


\begin{document}
\thispagestyle{empty}

\begin{center}
  \textbf{\Large <LEVEL> Test <NUMBER>}
  % LEVEL is Senior, Intermediate or Beginner
  % NUMBER is the test number: 1, 2, etc.
  \\ \vspace{1em}
  \textbf{\large Stellenbosch Camp 2022}
  \\ \vspace{1em}
  \textbf{\large Time: $2\frac{1}{2}$ hours}
\end{center}

\bigskip

\begin{enumerate}[itemsep=\fill]

\item % Source of problem
Problem statement


\item %


\item % Tsimerman
Can you tile a $10\times 10\times 10$ cube with $4\times 1\times 1$ blocks? (Standard tiling rules apply---cover the whole volume, no overlaps, etc.)

\textbf{Solution:} The answer is no. Consider the following colouring: there are 10 `layers' to the cube. Label these layers from 1 to 10. For layers 1,2,5,6,9,10, we colour the layer as follows:
\begin{center}
\includegraphics[scale=0.5]{Capture.png}
\end{center}
For layers 3,4,7,8, we invert these colours.\\
We note that each $4\times 1\times 1$ covers exactly 2 shaded blocks and 2 unshaded.\\
But counting the total number of shaded blocks gives 504, while there are 496 unshaded blocks.\\
Thus, we can not tile the shape as requested. 


\item % 


\item % 

\end{enumerate}


% ASCII art
\centering
\small
\begin{BVerbatim}
% Insert art here
\end{BVerbatim}

\end{document}

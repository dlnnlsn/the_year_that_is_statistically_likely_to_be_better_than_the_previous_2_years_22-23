\documentclass{article}

\usepackage{mathtools,amsfonts}
\usepackage{enumitem}
\usepackage{fullpage}
\usepackage{fancyvrb}
\usepackage{hyperref}


\begin{document}
\thispagestyle{empty}

\begin{center}
  \textbf{\Large Advanced Test 5 Solutions}
  \\ \vspace{1em}
  \textbf{\large Stellenbosch Camp 2022}
\end{center}

\bigskip

\begin{enumerate}[itemsep=24pt plus 12pt minus 9pt]

\item % Source of problem
Find all functions $f : \mathbb{R} \to \mathbb{R}$ such that \[ f(x+y) = f(x^2+y^2) \] for all $x,y \in \mathbb{R}$.

\textbf{Solution}:
We see that making the substitution $y \leftarrow -x$ yields 
\begin{align}
  f(0) = f(2x^{2}).
\end{align}
Consider any $a \in \mathbb{R}^+$, substitute $x$ $\rightarrow$ $\sqrt{\frac{a}{2}}$ into (1), this gives $f(0) = f(a)$ $\forall a \in \mathbb{R^{+}}$.

Substituting $x$ $\rightarrow$ $-x$ and $y$ $\rightarrow$ $0$ into the original equation gives $f(-x) = f((-x)^{2}) = f(x^{2})$.

Finally, substitute $y \rightarrow 0$ gives $f(x) = f(x^{2})$. Thus $f(x)= f(x^{2}) = f(-x)$ $\forall x \in \mathbb{R}$.

Therefore $f(x) = f(0) = c$ $\forall x \in \mathbb{R}$ and the check is trivial.


\item %
The game of Chomp is played on an $n \times n$ board by Dylan (on behalf of SA) and Fionn (on behalf of Ireland) as follows: Dylan moves first.
On a player's move, they must place an X on any square $(i, j)$ which does not yet have an X on it, and they must also fill in an X on any square above and to the right of that square which does not yet have an X on it.
That is, any square $(s, t)$ with $s \geq i$ and $t \geq j$ which does not yet have an X also gets an X filled in.
The person who places the last X loses.
Is there a winning strategy for either player?

\textbf{Solution:}
Dylan has a winning strategy that will absolutely demolish Fionn, leaving the whole nation of Ireland in tears.
Dylan's first move will be to place an $X$ on the square diagonally up-right from the bottom-left most square. As soon as Fionn sees this he will know he is doomed, he should honestly just resign now while he has some dignity remaining.
No matter what move Fionn makes now, Dylan will be able to mirror it on the opposing arm of the resulting L-shape, until a single square remains on Fionn's turn and he is sent back to Ireland as the loser.

The $n=1$ case doesn't count because that's the only case where Dylan loses which is unfair.


\item % Cono Sur Olympiad 1998
An acute triangle $\triangle ABC$ has circumcircle $\Gamma$ and orthocenter $H$.
$M$ is the midpoint of the segment $BC$.
Let $P$ be the point of the intersection of the line $HM$ with the minor arc $BC$ of $\Gamma$.
Let $Q \neq B$ be the intersection of the line $BH$ with $\Gamma$.
Prove that $PQ = BC$.

\textbf{Solution 1:  } Construct the line $k$ parallel to $HC$ through $B$, and the line $\ell$ parallel to $BH$ through $C$. Let $H'$ be the intersection of $k$ and $\ell$. Let the intersection of $AH$ with $BC$ be $D$ and the intersection of $BH$ with $AC$ be $E$. Notice that 
\begin{flalign*}
  \angle BH'C = \angle CH'B =180^\circ - \angle EHC =180^\circ - \angle EDC =180^\circ - \angle CAB
\end{flalign*}
Therefore, by the converse of opposite angles of a cyclic quadrilateral, $H', B, A, C$ are concyclic. In other words $H'$ is on the circumcircle of $\triangle ABC$.

Note that in parallelogram $HCH'B$ the diagonals bisect each other,so $M$ is on $H'H$. That is to say $H'$ is on $HM$, but $H'$ is also on the circumcircle so $H'=X$.

Finally, as $XC = H'C \parallel BH$ we have that $BYCX$ is an isosceles trapezium (cyclic + parallel sides), we have that $BC = XY$ as required.   

\textbf{Solution 2:} It is well known that $H$ is the center of the homothety of ratio 2 that sends the nine-point circle to the circumcircle. Therefore $M$ is the midpoint of $HX$, so $XC \parallel BH$ by midpoint theorem.
It follows that $BYCX$ is an isosceles trapezium, and thus $BC = XY$.


\item % Mario Ynocente Castro, Number Theory Marathon
Let $p$ be a prime and $a$, $b$, and $c$ be positive integers such that
\[ p = a+b+c-1 \quad \text{and} \quad p \mid a^3+b^3+c^3-1. \]
Prove that at least one of $a$, $b$, and $c$ is equal to $1$.

\textbf{Solution: } This is secretly an algebra question.
\begin{align*}
  a^3+b^3+c^3-1 &= (a+b+c-1)(a^2+b^2+c^2-ab-bc-ca)+3abc+a^2+b^2+c^2-ab-bc-ca-1\\
  &\equiv_p 3abc+a^2+b^2+c^2-ab-bc-ca-1\\
  &= (a+b+c-1)(a+b+c)+a+b+c+3abc-3ab-3bc-3ca-1\\
  &\equiv_p 3abc-3ab-3bc-3ca+a+b+c-1\\
  &\equiv_p 3abc-3ab-3bc-3ca+3a+3b+3c-3\\
  &= 3(a-1)(b-1)(c-1)
\end{align*}
If $p=3$, one of $a,b$ and $c$ must be $1$, else W.L.O.G. $p \mid (a-1)$ with $0\leq a-1<p$, which means $a=1$. 

\item % Parvardi, 100 Polynomial Problems
Find all polynomials $P(x)$ with real coefficients such that
\[ (x-8)P(2x) = 8(x-1)P(x) \]
for all real $x$.

\textbf{Solution:}
For the LHS to equal the RHS, the coefficients of the terms of the polynomial products must all be equal. Assuming $P(x) \neq 0$:
$$x \cdot a_n 2^nx^n = 8x\cdot a_n x^n \implies n = 3.$$
The roots of the LHS and RHS are also equal, meaning $8$ and $2$ are roots of $P(x)$.
\begin{align*}
  (x-8)c(2x-r_1)(2x-2)(2x-8) &= 8(x-1)c(x-r_1)(x-2)(x-8) \\
  \implies (2x-r_1)(x-4) &= 2(x-r_1)(x-2) \\
  \implies r_1 &= 4 \\
  \implies P(x) &= c(x-8)(x-4)(x-2) \qquad \forall c\in\mathbb{R}.
\end{align*}
The check is trivial.


\item % Russian MO, 2015, 10.6
A square is dissected into $n^2$ rectangles, where $n\geq 2$, by $n-1$ horizontal lines and $n-1$ vertical lines. Prove that one may choose $2n$ of these rectangles such that for any two of the chosen rectangles, one can be put completely into the other (perhaps, after some rotation).

\textbf{Solution:}
We can shuffle rows and columns, so assume W.L.O.G. that $a_1\geq ... \geq a_n$ are the lengths of the rectangles' horizontal sides (from the left to the right), and $b_1\geq ...\geq b_n$ are the lengths of their vertical sides (from the top to the bottom).
Let $Q_{i,j}$ be the $a_i\times b_j$ rectangle.
Since $a_1+...+a_n=b_1+...+b_n$, we cannot have $a_i<b_i$ for all $i$, or $b_i<a_i$ for all $i$.
Thus we have some $k$ such that $a_k\leq b_k$ and some $j$ such that $a_j\geq b_j$, and we also have that for every $i$, one of these is true.
This means that we can find an index $k$ such that $a_k\leq b_k$ and $a_{k-1}\geq b_{k-1}$ (or vice versa, but we consider this case without losing generality).
Then the required chain of rectangles is $Q_{1,1},Q_{1,2}...Q_{1,k-1},Q_{2,k-1},...,Q_{k,k-1},Q_{k-1,k},Q_{k,k},...,Q_{k,n},Q_{k+1,n},...,Q_{n,n}$.
It is clear that all $Q_{i,j}$ can be contained in $Q_{i,j+1}$ and also in $Q_{i+1,j}$.
And of course, our case gives that $Q_{k,k-1}$ can be contained in $Q_{k-1,k}$.


\end{enumerate}

\end{document}

\documentclass{article}

\usepackage{mathtools,amsfonts}
\usepackage{enumitem}
\usepackage[cm]{fullpage}
\usepackage{fancyvrb}
\usepackage{hyperref}
\usepackage{parskip}


\begin{document}
\thispagestyle{empty}

\begin{center}
  \textbf{\Large Advanced Test 2 Solutions}
  \\ \vspace{1em}
  \textbf{\large Stellenbosch Camp 2022}
\end{center}

\bigskip

\begin{enumerate}[itemsep=24pt]

\item % 2022, Emile
William and Beatrice take turns placing Kings on a $n \times m$ chessboard.
Kings cannot be placed on any of the 8 adjacent squares of Kings of \emph{differing} colour.
With William playing first as white, and Beatrice playing second as black, who has the winning strategy?

\textbf{Solution:}
William will always have a winning strategy. William's first move is to place his initial king on a square that either contains the center-point or has the center-point of the board on its border. William can then rotate Beatrice's moves $180^{\circ}$ through the center-point of the board. Note that Beatrice cannot place a king in such a way that the reflection of her move is within the $8$ adjacent squares of the placed king. Therefor William will always be able to make a move as any legal move Beatrice plays will also have to be legal for William to play on the reflection. As the game is forced to end in less than $nm$ moves, William will place the last piece and have the winning strategy.


\item % Baltic Way 1994
In $\triangle ABC$ let $\angle C = 90^\circ$, and let $\Gamma$ be the circle with diameter $AC$. Define points $D$ and $E$ on $\Gamma$ such that $D$ is on $AB$ and $DE \parallel AC$. Let $P$ be the intersection of $AE$ and $BC$. Prove that
\[ PC \cdot BC = AC^2. \]

\textbf{Solution:} Notice that the statement is true if $\triangle ABC \operatorname{|||} \triangle PAC$, but $\angle PCB = \angle ACB$ so the problem is reduced to showing that $\angle ABC = \angle PAC$. We have that
\begin{flalign*}
&& \angle ABC &= 90^\circ - \angle CAB  & (\angle's \text{ in } \triangle)\\
&& &= 90^\circ - (180^\circ - \angle ADE)  & (\text{co-int } \angle's)\\
&& &= 90^\circ - \angle ECA  & (\text{opp } \angle's\text{ cyc. quad}).
\end{flalign*}
But $AC$ is the diameter of $\Gamma$ so $\angle AEC = 90^\circ$ by Thales' Theorem. Therefore
\begin{flalign*}
  \angle ABC = 90^\circ - \angle ECA = \angle CAE = \angle PAC.
\end{flalign*}


\item %
The given recurrence relation is equivalent to
\[
    \frac{1}{a_{n}} = \frac{1}{a_{n - 1}} + \frac{1}{a_{n - 2}}.
\]
It follows that (e.g. using induction)
\[
    \frac{1}{a_{n}} = \frac{F_{n - 2}}{\ell} + \frac{F_{n - 1}}{m},
\]
where $F_k$ is the $k^\text{th}$ Fibonacci number.

We thus have that
\[
    a_{2022} = \frac{\ell m}{m F_{2020} + \ell F_{2021}}.
\]

We note that for any natural number $k$, we have that $\gcd(F_k, F_{k + 1}) = 1$. (This can also be proven using induction.) By B\'ezout's Lemma, we know that there are infinitely many integers $\ell$ and $m$ such that $\ell F_{2020} + m F_{2021} = \gcd(F_{2020}, F_{2021}) = 1.$

It is clear that one of $\ell$ and $m$ is positive, and the other is negative. For these values of $\ell$ and $m$ we have that
\[
    \frac{(-\ell) (-m)}{(-m) F_{2020} + (-\ell) F_{2021}} = \frac{\ell m}{-1}
\]
which is a positive integer. (Here we take $a_1 = -\ell$ and $a_2 = -m$.)


\item % M Ahsan Al Mahir, 2020
Evaluate the following expression for all positive integers $n$:
\[ {2n \choose 0} -{2n-1 \choose 1}+{2n-2 \choose 2}-...+(-1)^n{n \choose n} \]

\textbf{Solution:} Define:
\[ E_n = {2n \choose 0} -{2n-1 \choose 1}+{2n-2 \choose 2}-...+(-1)^n{n \choose n} \ \text{and} \ O_n = {2n+1 \choose 0} -{2n \choose 1}+{2n-1 \choose 2}-...+(-1)^n{n+1 \choose n}. \]
Then, using the fact that ${m\choose k} = {m-1 \choose k} + {m-1 \choose k-1}$:
\begin{align*}
E_n &= \begin{multlined}[t] {2n-1 \choose 0} - \left({2n-2 \choose 1} + {2n-2 \choose 0}\right) + \left({2n-3 \choose 2} + {2n-3 \choose 1}\right) - ...\\ + (-1)^{n-1}\left({n \choose n-1} + {n \choose n-2}\right)+(-1)^n {n-1 \choose n-1} \end{multlined} \\
&= \begin{multlined}[t] \left({2n-1 \choose 0}-{2n-2 \choose 1}+{2n-3\choose 2}-...+(-1)^{n-1}{n\choose n-1}\right) \\ - \left({2n-2\choose 0} - {2n-3\choose 1}+...+(-1)^{n-1}{n\choose n-2} + (-1)^n{n-1 \choose n-1}\right) \end{multlined} \\
&= O_{n-1}-E_{n-1}
\end{align*}
A similar argument shows that $O_n = E_n - O_{n-1}$. Then:
\[ E_n = O_{n-1}-E_{n-1} = E_{n-1}-O_{n-2}-E_{n-1} = -O_{n-2} = -(E_{n-2}-O_{n-3}) = -(O_{n-3}-E_{n-3}-O_{n-3}) = E_{n-3}. \]
Now, since $E_0 = 1, E_1 = 0, E_2 = -1$, we have:
$$E_i = \begin{cases}
1 &\quad\text{if } i \bmod 3 = 0 \\
0 &\quad\text{if } i \bmod 3 = 1 \\
-1 &\quad\text{if } i \bmod 3 = 2 \\
\end{cases}.$$


\item % PAMO 2016 P4
Let $x$, $y$, and $z$ be positive real numbers such that $xyz = 1$.
Prove that
\[ \frac{x^2y^2}{y^2(x+1)^2+x^2+x^2y^2} +\frac{y^2z^2}{z^2(y+1)^2+y^2+y^2z^2} +\frac{z^2x^2}{x^2(z+1)^2+z^2+z^2x^2} \leq \frac{1}{2}. \]

\textbf{Solution:}
Since $xyz = 1$, we can let $x = a/b$, $y = b/c$, and $z = c/a$ where $a,b,c > 0$.
Then
\begin{align*}
  \frac{x^2y^2}{y^2(x+1)^2+x^2+x^2y^2} &= \frac{a^2/c^2}{(a+b)^2/c^2+a^2/b^2+a^2/c^2} = \frac{a^2b^2}{(a+b)^2b^2+a^2c^2+a^2b^2} = \frac{a^2b^2}{2a^2b^2+2ab^3+b^4+a^2c^2} \\
  &\leq \frac{a^2b^2}{2a^2b^2+2ab^3+2ab^2c} = \frac{a}{2a+2b+2c} \quad\text{since}\ b^4+a^2c^2 \geq 2ab^2c \ \text{by AM-GM.}
\end{align*}
Similarly, for the other two terms we get
\[ \frac{y^2z^2}{z^2(y+1)^2+y^2+y^2z^2} \leq \frac{b}{2a+2b+2c} \quad\text{and}\quad \frac{z^2x^2}{x^2(z+1)^2+z^2+z^2x^2} \leq \frac{c}{2a+2b+2c}, \]
and so adding these together we get
\begin{align*}
  &\mspace{24mu} \frac{x^2y^2}{y^2(x+1)^2+x^2+x^2y^2} +\frac{y^2z^2}{z^2(y+1)^2+y^2+y^2z^2} +\frac{z^2x^2}{x^2(z+1)^2+z^2+z^2x^2} \\
  &\leq \frac{a}{2a+2b+2c} +\frac{b}{2a+2b+2c} +\frac{c}{2a+2b+2c} = \frac{1}{2}
\end{align*}
as desired.

\end{enumerate}

\end{document}
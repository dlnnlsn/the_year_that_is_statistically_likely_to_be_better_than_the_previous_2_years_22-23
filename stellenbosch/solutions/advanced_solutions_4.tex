\documentclass{article}

\usepackage{mathtools,amsfonts}
\usepackage{enumitem}
\usepackage{fullpage}
\usepackage{fancyvrb}
\usepackage{hyperref}
\usepackage{parskip}


\begin{document}
\thispagestyle{empty}

\begin{center}
  \textbf{\Large Advanced Test 4 Solutions}
  \\ \vspace{1em}
  \textbf{\large Stellenbosch Camp 2022}
  \\ \vspace{1em}
\end{center}

\bigskip \bigskip

\begin{enumerate}[itemsep=24pt]

\item % Emile, 2022
You have an $n \times n$ chessboard which starts with all its squares white.
An operation involves choosing a row or column and flipping the colour of every square in this row or column between white and black.
Is it possible that for all $0 \leq k \leq n^2$, you can do some sequence of operations and end up with $k$ black squares?

\textbf{Solution:}
First note that performing a flip on a row or column twice is the same as having not performed a flip at all. Secondly note that the order of flips do not matter. Let there be $p$ row-flips and $q$ column-flips, with $0\leq p,q\leq n$. The number of values of $k$ we can achieve with a series of operations is less than or equal to the number of ordered pairs $(p,q)$, which we will define as $S_n$. The ordered restriction is due to $(p,q)$ and $(q,p)$ yielding the same number of black squares. If $S_n$ is less than the number of squares on the board, then there will exist some number of $k$ black squares we cannot achieve. Therefore we need that:
$$S_n = \frac{(n+1)(n+2)}{2} \geq n^2 + 1$$
Which is only true for $n\leq3$. Checking the number of black squares for these cases with all the possible pairs $(p,q)$ shows that only the $n=1$ case is possible.


\item %
Show that $2022 \times 2^n$ can be written as the sum of three distinct non-zero squares for every natural number $n$.

\textbf{Solution:}
Note that $1011 = 31^2 + 7^2 + 1^2$ and $2022 = 43^2 + 13^2 + 2^2$.

If $n$ is odd, let $n = 2k + 1$. Then we have that
\[
    2022 \times 2^n = 1011 \times 2^{2k + 2} = \left( 31^2 + 7^2 + 1^2 \right) \times {\left( 2^{k + 1} \right)}^2 = {\left( 31 \times 2^{k + 1} \right)}^2 + {\left( 7 \times 2^{k + 1} \right)}^2 + {\left( 1 \times 2^{k + 1} \right)}^2.
\]

Similarly, if $n$ is even, let $n = 2k$. Then we have that
\[
    2022 \times 2^n = 2022 \times 2^{2k} = \left( 43^2 + 13^2 + 2^2 \right) \times {\left( 2^{k} \right)}^2 = {\left( 43 \times 2^{k} \right)}^2 + {\left( 13 \times 2^{k} \right)}^2 + {\left( 2 \times 2^{k} \right)}^2.
\]


\item % Andrew McGregor, 2022.
A triple of positive real numbers $(a,b,c)$ is called \textit{good} if $a,b,c$ are the side lengths of a non-degenerate triangle, and \textit{cute} if $a,b,c$ are the sides of an acute angled triangle. Prove that $(a,b,c)$ is \textit{good} if and only if $(\sqrt{a},\sqrt{b},\sqrt{c})$ is \textit{cute}. 

\textbf{Solution:}
Notice that if $(\sqrt{a},\sqrt{b},\sqrt{c})$ is \textit{cute}, then $\sqrt{a}^2 + \sqrt{b}^2 > \sqrt{c}^2$ for all permutatios of $(a,b,c)$ (Pythagoras' inequality) so $(\sqrt{a}^2,\sqrt{b}^2,\sqrt{c}^2)=(a,b,c)$ is \textit{good}.

If $(a,b,c)$ is good we note that it is sufficient to prove that $(\sqrt{a},\sqrt{b},\sqrt{c})$ is \textit{good} as Pythagoras' inequality would be trivialy satisfied. Use the Ravi substitution $a=x+y,b=y+z,c=z+x$ where $x,y,z>0$. Then
\begin{flalign*}
  \sqrt{a}+\sqrt{b} &= \sqrt{x+y} +\sqrt{y+z}\\
  &= \sqrt{\left( \sqrt{x+y} +\sqrt{y+z} \right)^2 }\\
  &= \sqrt{\left( x+y +y+z + 2\sqrt{(x+y)(y+z)}\right) }\\
  &= \sqrt{\left( x+z + (2y+ 2\sqrt{(x+y)(y+z)})\right) }\\
  &> \sqrt{x+z}\\
  &= \sqrt{c}
\end{flalign*}
Similarly, $\sqrt{a}+\sqrt{c} >\sqrt{b}$ and $\sqrt{b}+\sqrt{c} >\sqrt{a}$.



\item % % Baltic Way 1994
The incircle of $\triangle ABC$ is tangent to $BC,CA,$ and $AB$ at $D,E,$ and $F$ respectively.
$P$, $Q$, and $R$ are the incenters of $\triangle AEF$, $\triangle BFD$, and $\triangle CDE$ respectively.
Prove that $DP$, $EQ$, and $RS$ are concurrent.

\textbf{Solution:}
Let the point $P'$ be the intersection of $AI$ and the incircle. Notice that $\angle AFP' = \angle P'EF = \angle P'FE$ (tan-chord, angles subtended by equal sides) therefore $FP'$ is the angle bisector of $\angle AFE$, so $P'=P$ is the incentre of $\triangle AFE$.
Similarly we deduce $Q$ and $R$ are the intersections of $BI$ and $CI$ with the incircle respectively.
Then $\angle PDF = \angle PDE$ (angles subtended by equal sides) so $DP$ is the angle bisector of $\angle FDE$.
Similarly we deduce that $EQ$ and $FR$ are angle bisectors of $\angle DEF$ and $\angle EFD$ respectively.
The three lines intersect at the incenter of $\triangle DEF$ and are therefore concurrent.


\item % Tsimelman, 2022
Every point in space is coloured red, blue, or green. Prove that for at least one of the 3 colours, the set of distances between points of that colour contains all positive real numbers.

\textbf{Solution:}
Assume for contradiction that none of the colours has this property.
Let $x_R$ be a positive real number for which there are no two red points having this distance between them.
Similarly, let $x_B$ and $x_G$ be such a real number for the colours blue and green respectively.
Assume W.L.O.G. that $x_R\geq x_B \geq x_G$.
Start by selecting a red point.
If no such point exists, select a blue point.
If no such point exists, we only have green points and we reach a clear contradiction.
If a blue point does exist (but a red point does not), all points in a shell of radius $x_B$ around that point must be green.
Then we have two green points on this shell with distance $x_G$ between them, since $x_G\leq x_B$, which is a contradiction.

Finally, if our red point does exist, every point in a shell of radius $x_R$ around it must be either blue or green.
Select a blue point on this shell.
If no such point exists, this whole shell is green and there must exist two points on this shell with distance $x_G$ between them, a contradiction.
If the blue point does exist, we consider the intersection of the shell around our blue point with radius $x_B$ and the shell around our red point with radius $x_R$.
This intersection is a circle with radius at least $\frac{x_B}{\sqrt{2}}$, which means it has diameter larger than $x_B$.
All points on this circle must be green, but there will exist two of them with distance $x_G$, a contradiction.

\end{enumerate}

\end{document}

\documentclass{article}

\usepackage{mathtools,amsfonts}
\usepackage{enumitem}
\usepackage{fullpage}
\usepackage{fancyvrb}
\usepackage{hyperref}


\begin{document}
\thispagestyle{empty}

\begin{center}
  \textbf{Advanced Test 2 Solutions}
  % LEVEL is Senior, Intermediate or Beginner
  % NUMBER is the test number: 1, 2, etc.
  \\ \vspace{1em}
  \textbf{\large Stellenbosch Camp 2022}
\end{center}


\begin{enumerate}[itemsep=\fill]

\item % Source of problem
Problem statement


\item % 2022, Andrew
Find all functions $f:\mathbb{R}\to\mathbb{R}$ such that:
\[f(x) + f(y) = f(2x+y) - x\] for all $x,y\in\mathbb{R}$.

\textbf{Solution:} Let $y = -x$:
\[f(x) + f(-x) = f(2x-x) - x\]
\[f(-x) = -x\implies f(x) = x\]
The check is trivial.

\item 
Find all $m,n\in \mathbb{Z}$ satisfying the following equation:
\begin{align}
m^3 + n^3 = (m+n)^2 
\end{align}

\textbf{Solution:} Notice that $m^3 + n^3$ can be factorised as $m^3 + n^3=(m+n)(m^{2} -mn + n^{2})$, thus equation (1) becomes:
\begin{align*}
    (m+n)(m^{2} -mn + n^{2}) & = (m+n)^2\\
    (m+n)(m^{2} -mn + n^{2} -m -n) & = 0 \\
\end{align*}
Thus, either $m = -n$ or $m^{2} -mn + n^{2} -m -n = 0$. The former is clearly a solution, so we consider the latter case henceforth. Observe that one can make the following factorisation:
\begin{align}
    m^{2} -mn + n^{2} -m -n & = 0 \nonumber\\ 
    m^{2} - mn + \frac{n^{2}}{4} + \frac{3n^{2}}{4} - m -n & = 0 \nonumber\\
    \left(m - \frac{n}{2}\right)^{2} - \left(m - \frac{n}{2}\right) + \frac{3n^{2}}{4} - \frac{3n}{2} & = 0 \nonumber\\
    \left(m - \frac{n}{2}\right)^{2} - \left(m - \frac{n}{2}\right) + \frac{1}{4} + \frac{3n^{2}}{4} - \frac{3n}{2} + \frac{3}{4} & = \frac{1}{4} +\frac{3}{4} \nonumber\\
    \left(m - \frac{n}{2} - \frac{1}{2}\right)^{2} +3\left(\frac{n}{2} - \frac{1}{2}\right)^{2} & = 1 \nonumber\\
    (2m - n - 1)^{2} + 3(n-1)^{2} & = 4 
\end{align}

If $(n-1)^{2} \geq 2$, we have
\begin{align*}
    3(n-1)^{2} & \geq 6 \\
    (2m - n - 1)^{2} & \geq 0 \\
    \Rightarrow 4 = (2m - n - 1)^{2} + 3(n-1)^{2} & \geq 6
\end{align*}
Which is not possible, thus $(n-1)^{2} \leq 1$. So we have either $(n-1)^{2} = 1$ or $(n-1)^{2} = 0$
\begin{enumerate}
\item If $(n-1)^{2} = 1$: $n = 2$ or $n=0$.
\begin{enumerate}
    \item If $n=2$, then $(2)$ becomes $(2m -3)^{2} + 3 = 4$ $\Rightarrow$ $2m - 3 = 1$ or $2m - 3 = -1$. Thus $(m, n) = (2, 2), (1, 2)$. \newline
    \item If $n = 0$, $(2)$ becomes $(2m -1)^{2} + 3 = 4$. Thus $\Rightarrow$ $2m- 1 = 1$ or $2m - 1 = -1$. Yielding $(m, n) = (1, 0), (0, 0)$
\end{enumerate}
\item If $(n-1)^{2} = 0$: $n = 1$, then (2) becomes $(2m - 2)^{2} = 4$. Thus $(m, n) = (2, 1), (0, 1)$
\end{enumerate}

\item % 2022, Emile
William will always have a winning strategy. William's first move is to place his initial king on a square that either contains the center-point or has the center-point of the board on its border. William can then rotate Beatrice's moves $180^{\circ}$ through the center-point of the board. Note that Beatrice cannot place a king in such a way that the reflection of her move is within the $8$ adjacent squares of the placed king. Therefor William will always be able to make a move as any legal move Beatrice plays will also have to be legal for William to play on the reflection. As the game is forced to end in less than $nm$ moves, William will place the last piece and have the winning strategy.


\item % Baltic Way 1994
In $\triangle ABC$ let $\angle C = 90^\circ$, and let $\Gamma$ be the circle with diameter $AC$. Define points $D$ and $E$ on $\Gamma$ such that $D$ is on $BC$ and $DE || AC$. Let $P$ be the intersection of $AE$ and $BC$. Prove

\begin{flalign*}
  PC \cdot BC = AC^2.
\end{flalign*}

\textbf{Solution:} Notice that the statement is true if $\triangle ABC ||| \triangle PAC$, but $\angle PCB = \angle ACB$ so the problem is reduced to showing that $\angle ABC = \angle PAC$. We have that
\begin{flalign*}
&& \angle ABC &= 90^\circ - \angle CAB  & (\angle's \text{ in } \triangle)\\
&& &= 90^\circ - (180^\circ - \angle ADE)  & (\text{co-int } \angle's)\\
&& &= 90^\circ - \angle ECA  & (\text{opp } \angle's\text{ cyc. quad}).
\end{flalign*}
But $AC$ is the diameter of $\Gamma$ so $\angle AEC = 90^\circ$ by Thales' Theorem. Therefore
\begin{flalign*}
  \angle ABC = 90^\circ - \angle ECA = \angle CAE = \angle PAC.
\end{flalign*}

\end{enumerate}


% ASCII art
\centering
\small
\begin{BVerbatim}
% Insert art here
\end{BVerbatim}

\end{document}

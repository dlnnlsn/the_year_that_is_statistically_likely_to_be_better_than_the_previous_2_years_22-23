\documentclass{article}

\usepackage{mathtools,amsfonts}
\usepackage{enumitem}
\usepackage{fullpage}
\usepackage{fancyvrb}
\usepackage{hyperref}


\begin{document}
\thispagestyle{empty}

\begin{center}
  \textbf{\Large <LEVEL> Test <NUMBER>}
  % LEVEL is Senior, Intermediate or Beginner
  % NUMBER is the test number: 1, 2, etc.
  \\ \vspace{1em}
  \textbf{\large Stellenbosch Camp 2022}
  \\ \vspace{1em}
  \textbf{\large Time: $2\frac{1}{2}$ hours}
\end{center}

\bigskip

\begin{enumerate}[itemsep=\fill]

\item % Source of problem
Problem statement


\item %


\item % 


\item % 


\item % Baltic Way 1994
In $\triangle ABC$ let $\angle C = 90^\circ$, and let $\Gamma$ be the circle with diameter $AC$. Define points $D$ and $E$ on $\Gamma$ such that $D$ is on $BC$ and $DE || AC$. Let $P$ be the intersection of $AE$ and $BC$. Prove

\begin{flalign*}
  PC \cdot BC = AC^2.
\end{flalign*}

\textbf{Solution:} Notice that the statement is true if $\triangle ABC ||| \triangle PAC$, but $\angle PCB = \angle ACB$ so the problem is reduced to showing that $\angle ABC = \angle PAC$. We have that
\begin{flalign*}
&& \angle ABC &= 90^\circ - \angle CAB  & (\angle's \text{ in } \triangle)\\
&& &= 90^\circ - (180^\circ - \angle ADE)  & (\text{co-int } \angle's)\\
&& &= 90^\circ - \angle ECA  & (\text{opp } \angle's\text{ cyc. quad}).
\end{flalign*}
But $AC$ is the diameter of $\Gamma$ so $\angle AEC = 90^\circ$ by Thales' Theorem. Therefore
\begin{flalign*}
  \angle ABC = 90^\circ - \angle ECA = \angle CAE = \angle PAC.
\end{flalign*}

\end{enumerate}


% ASCII art
\centering
\small
\begin{BVerbatim}
% Insert art here
\end{BVerbatim}

\end{document}

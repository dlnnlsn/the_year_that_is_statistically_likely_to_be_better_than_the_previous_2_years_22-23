\documentclass{article}

\usepackage{mathtools,amsfonts}
\usepackage{enumitem}
\usepackage{fullpage}
\usepackage{fancyvrb}
\usepackage{hyperref}


\begin{document}
\thispagestyle{empty}

\begin{center}
  \textbf{\Large Advanced Test 1}
  \\ \vspace{1em}
  \textbf{\large Stellenbosch Camp 2022}
  \\ \vspace{1em}
  \textbf{\large Time: $2\frac{1}{2}$ hours}
\end{center}

\bigskip

\begin{enumerate}[itemsep=\fill]


\item % Tim, 2022
There are at least 3 people at a party. All of them have an even number of friends, where friendship is mutual. Show that there are 3 of them who each have the same number of friends.

\textbf{Solution:} \item If there are $2n$ people, then they can have $0,2,...$ or $2n-2$ friends. There are thus $n$ options for how many friends each person can have. If there is someone with 0 friends, there cannot be 2 people with $2n-2$ friends and if there is someone with $2n-2$ friends there cannot be 2 people with 0 friends. We now consider 2 cases:
\begin{enumerate}
\item There is a number of friends that no one has (either 0 or $2n-2$). Then there are $n-1$ options for how many friends people have, and $2n$ people. Thus, there must be 3 people that have the same number of friends (by pigeonhole principle).
\item There is one person who has 0 friends and one person who has $2n-2$. Then there are $2n-2$ other people, and $n-2$ options for how many friends they have. Again, there must now be 3 people that have the same number of friends.
\end{enumerate}
If instead there are $2n+1$ people, they can have $0,2,...$ or $2n$ friends. There are thus $n+1$ options for how many friends each person can have. If someone has 0 friends, no one can have $2n$. Then we have $n$ options with $2n+1$ people, so there must be 3 people that have the same number of friends.


\item % Malwande, 2022
Let $ABC$ be an acute-angled triangle. Let $D$, $E$, and $F$ be the feet of the perpendiculars from $A$, $B$, and $C$ onto $BC$, $CA$, and $AB$ respectively. The incircle of triangle $DEF$ touches $EF$, and $DF$ at $X$ and $Y$ respectively. Prove that $XY$ is parallel to $AB$.

\textbf{Solution:} Notice that the orthocenter, $H$, of $\triangle ABC$ is the in incenter of $\triangle DEF$. Note that $XF=YF$ (common tangents), therefore $FH \perp XY$ (angle bisector of isosceles $\triangle\ \perp base$), but $FH \perp AB$ as $H$ is the orthocenter, therefore, $AB || XY$ (corresponding angles).

\item % % Kgaugelo, 2022
Given positive real numbers $a$, $b$, and $c$ such that $a+b+c = 3$ and $a^2+b^2+c^2 = 3$. Find the value of \[\frac{a}{b} +\frac{b}{c} +\frac{c}{a}.\]

\textbf{Solution:} We claim that $a=b=c=1$ so that $\frac{a}{b} +\frac{b}{c} +\frac{c}{a} = 3$. 
\\ Squaring yields 
\begin{align*}
	a+b+c &= 9
	\\ \implies (a+b+c)^2 &= 3
	\\ \implies a^2+b^2+c^2 + 2ab + 2bc + 2ca &= 9
	\\ \implies ab + bc + ca &= 3
\end{align*}
Now suppose WLOG that $a \geq b \geq c$ so that the rearrangement inequality gives that
$$a \cdot a + b \cdot b + c \cdot c = a^2+b^2+c^2 \geq ab + bc +ca$$
with equality holding if and only if $a=b=c$. 
Thus $a=b=c=1$ as required. 

\item %


\end{enumerate}


% ASCII art
\centering
\small
\begin{BVerbatim}
% Insert art here
\end{BVerbatim}

\end{document}

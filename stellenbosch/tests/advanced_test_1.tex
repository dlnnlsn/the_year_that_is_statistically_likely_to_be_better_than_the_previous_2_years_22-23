\documentclass{article}

\usepackage{mathtools,amsfonts}
\usepackage{enumitem}
\usepackage{fullpage}
\usepackage{fancyvrb}
\usepackage{hyperref}


\begin{document}
\thispagestyle{empty}

\begin{center}
  \textbf{\Large Advanced Test 1}
  \\ \vspace{1em}
  \textbf{\large Stellenbosch Camp 2022}
  \\ \vspace{1em}
  \textbf{\large Time: $2\frac{1}{2}$ hours}
\end{center}

\bigskip

\begin{enumerate}[itemsep=\fill]

\item % Liam, 2022
Where $p$ is a prime and $n\in\mathbf{N}$, find all solutions to the equation \[p^2 = 2^n + 1.\]

\vspace{0pt}

\item % Tim, 2022
There are at least 3 people at a party. All of them have an even number of friends, where friendship is mutual. Show that there are 3 of them who each have the same number of friends.


\item % Malwande, 2022
Let $ABC$ be an acute-angled triangle. Let $D$, $E$, and $F$ be the feet of the perpendiculars from $A$, $B$, and $C$ onto $BC$, $CA$, and $AB$ respectively. The incircle of triangle $DEF$ touches $EF$, and $DF$ at $X$ and $Y$ respectively. Prove that $XY$ is parallel to $AB$.


\item % 1,1,3,8,21,57


\item % Tim, 2022
For every positive integer $n$, define \[ r(n) = (n \bmod 1) + (n \bmod 2) + \dotsb + (n \bmod n), \]
where by $n \bmod m$ we denote the remainder of $n$ on division by $m$.
Show that there are infinitely many positive integers $k$ such that $r(k) = r(k+1)$.


\end{enumerate}


\vfill
% ASCII art
\centering
\small
\begin{BVerbatim}
% Insert art here
\end{BVerbatim}

\end{document}

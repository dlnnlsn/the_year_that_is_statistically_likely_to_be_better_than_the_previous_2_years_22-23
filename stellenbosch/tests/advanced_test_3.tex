\documentclass{article}

\usepackage{mathtools,amsfonts}
\usepackage{enumitem}
\usepackage{fullpage}
\usepackage{fancyvrb}
\usepackage{hyperref}


\begin{document}
\thispagestyle{empty}

\begin{center}
  \textbf{\Large Advanced Test 3}
  % LEVEL is Senior, Intermediate or Beginner
  % NUMBER is the test number: 1, 2, etc.
  \\ \vspace{1em}
  \textbf{\large Stellenbosch Camp 2022}
  \\ \vspace{1em}
  \textbf{\large Time: $2\frac{1}{2}$ hours}
\end{center}

\bigskip

\begin{enumerate}[itemsep=\fill]

\item % IMO Longlist 1983
For acute triangle $\triangle ABC$, a point $Z$ interior to $\triangle ABC$ satisfies $ZB=ZC$. Suppose points $X$ and $Y$ lie outside $\triangle ABC$ such that $\triangle XAB |||\triangle YCA |||\triangle ZBC$. Prove that $A,X,Y,Z$ are the vertices of a parallelogram.

\item %


\item %Tsimerman
Can you tile a $10\times 10\times 10$ cube with $4\times 1\times 1$ blocks? (Standard tiling rules apply - cover the whole volume, no overlaps, etc.)


\item % 
Consider an infinite sequence $(a_n)_{n=0}^{\infty}$ of positive integers such that for $n \geq 0$, we have that
\[
    a_{n + 1} = a_n + b_n
\]
where $b_n$ is the last (units) decimal digit of $a_n$. Show that the sequence contains infinitely many powers of $2$ if and only if $a_0$ is not a multiple of $5$.


\item %  IMO Shortlist 1998 G3
Let $I$ be the incenter of triangle $\triangle ABC$. Let $D,E$ and $F$ be the points of tangency of the incircle of $\triangle ABC$ with $BC,CA$ and $AB$, respectively. The line $\ell$ passes through $B$ and is parallel to $DF$. The lines $ED$ and $EF$ intersect $\ell$ at the points $X$ and $Y$. Prove that $\angle XIY$ is acute.
\end{enumerate}


% ASCII art
\centering
\small
\begin{BVerbatim}
% Insert art here
\end{BVerbatim}

\end{document}

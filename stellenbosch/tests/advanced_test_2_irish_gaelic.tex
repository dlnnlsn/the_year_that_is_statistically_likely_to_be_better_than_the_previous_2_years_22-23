\documentclass{article}

\usepackage{mathtools,amsfonts}
\usepackage{enumitem}
\usepackage{fullpage}
\usepackage{fancyvrb}
\usepackage{hyperref}


\begin{document}
\thispagestyle{empty}

\begin{center}
  \textbf{\Large Scrúdú 2}
  % LEVEL is Senior, Intermediate or Beginner
  % NUMBER is the test number: 1, 2, etc.
  \\ \vspace{1em}
  \textbf{\large Campa traenála i Stellenbosch 2022}
  \\ \vspace{1em}
  \textbf{\large Teorainn ama: $2\frac{1}{2}$ uair an chloig}
\end{center}

\bigskip

\vfill

\begin{enumerate}[itemsep=\fill]

\item % 2022, Emile
Glacann William agus Beatrice seal ag cur Kings ar chlár fichille $n \times m$.
Ní féidir ríthe a chur ar aon cheann de na 8 gcearnóg in aice láimhe de dhathanna Ríthe \emph{difriúil}.
Le William ag imirt sa chéad áit chomh bán, agus Beatrice ag imirt sa dara háit le dubh, cé aige a bhfuil an straitéis bhuaiteach?


\item % Baltic Way 1994
I $\triangle ABC$ lig $\angle C = 90^\circ$, agus lig $\Gamma$ mar an ciorcal dar trastomhas $AC$.
Sainmhínigh pointí $D$ agus $E$ ar $\Gamma$ ionas go mbeidh $D$ ar $AB$ agus $DE \parallel AC$.
Bíodh $P$ mar a dtrasnaíonn $AE$ agus $BC$.
Cruthaigh sin
\[ PC \cdot BC = AC^2. \]

\vspace{0pt}


\item % Malwande, 2022
Bíodh $a_1, a_2, a_3, \dots$ mar sheicheamh uimhreacha arna sainmhíniú ag
\begin{itemize}
    \item $a_1 = l$
    \item $a_2 = m$
    \item $a_n = \dfrac{a_{n-1}a_{n-2}}{a_{n-1}+a_{n-2}}$ do gach slánuimhir $n \geq 3$.
\end{itemize}
Cruthaigh gur iomaí péire slánuimhreacha $l$ agus $m$ gan teorainn sa chaoi is gur slánuimhir dhearfach é $a_{2022}$.


\item % M Ahsan Al Mahir, 2020
Déan an slonn seo a leanas a mheas do gach slánuimhir dheimhneach $n$:
\[ {2n \choose 0} -{2n-1 \choose 1}+{2n-2 \choose 2}-...+(-1)^n{n\choose n} \]


\vspace{0pt}


\item % PAMO 2016 P4
Bíodh $x$, $y$, agus $z$ ina bhfíoruimhreacha dearfacha ionas go mbeidh $xyz = 1$.
Cruthaigh sin
\[ \frac{x^2y^2}{y^2(x+1)^2+x^2+x^2y^2} +\frac{y^2z^2}{z^2(y+1)^2+y^2+y^2z^2} +\frac{z^2x^2}{x^2(z+1)^2+z^2+z^2x^2} \leq \frac{1}{2}. \]

\end{enumerate}


\vfill
% ASCII art
\centering
\small
\begin{BVerbatim}


    .--.              .--.
   : (\ ". _......_ ." /) :
    '.    `        `    .'
     /'   _        _   `\
    /     0}      {0     \
   |       /      \       |
   |     /'        `\     |
    \   | .  .==.  . |   /
     '._ \.' \__/ './ _.'
     /  ``'._-''-_.'``  \


\end{BVerbatim}

\end{document}

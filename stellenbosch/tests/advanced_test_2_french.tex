\documentclass{article}

\usepackage{mathtools,amsfonts}
\usepackage{enumitem}
\usepackage{fullpage}
\usepackage{fancyvrb}
\usepackage{hyperref}


\begin{document}
\thispagestyle{empty}

\begin{center}
  \textbf{\Large Test Avanc\'e 2}
  % LEVEL is Senior, Intermediate or Beginner
  % NUMBER is the test number: 1, 2, etc.
  \\ \vspace{1em}
  \textbf{\large Camp Stellenbosch 2022}
  \\ \vspace{1em}
  \textbf{\large Temps: $2\frac{1}{2}$ heures}
\end{center}

\bigskip

\vfill

\begin{enumerate}[itemsep=\fill]

\item % 2022, Emile
William et Beatrice jouent \`a un jeu. Ils placent des rois sur un \'echiquier $n \times m$. Les rois ne peuvent pas \^etre plac\'es sur une case qui partage un point avec une case qui contient un roi d'une couleur diff\'erente. William joue en premier et place les rois blancs. Beatrice joue deuxi\`eme et place les rois noirs. Y a-t-il une stat\'egie gagnant et qui a la strat\'egie ?


\item % Baltic Way 1994
Soit $\triangle ABC$ un triangle avec $\angle C = 90^\circ$, et soit $\Gamma$ le cercle de diam\`etre $AC$.
Soient $D$ et $E$ des points sur $\Gamma$ tels que $D$ se trouve en $AB$ et $DE \parallel AC$. Soit $P$ le point d'intersection des droites $AE$ et $BC$.
Montrer que
\[ PC \cdot BC = AC^2. \]

\vspace{0pt}


\item % Malwande, 2022
Soient $a_1, a_2, a_3, \dots, a_n$ une suite de nombres r\'eels d\'efinie par:
\begin{itemize}
	\item $a_1 = l$
	\item $a_2 = m$
	\item $a_n = \dfrac{a_{n-1}a_{n-2}}{a_{n-1}+a_{n-2}}$ pour tout les entiers $n \geq 3$.
\end{itemize}
Montrer qu'il existe une infinit\'e de pairs d'entiers $l$ et $m$ tels que $a_{2022}$ soit un entier strictement positif.


\item % M Ahsan Al Mahir, 2020
Trouver la valeur de l'expression suivant pour chaque entier strictment positif $n$:
\[ {2n \choose 0} -{2n-1 \choose 1}+{2n-2 \choose 2}-...+(-1)^n{n \choose n} \]

\vspace{0pt}


\item % PAMO 2016 P4
Soient $x$, $y$, et $z$ des r\'eels strictement positifs tels que $xyz = 1$. Montrer que
\[ \frac{x^2y^2}{y^2(x+1)^2+x^2+x^2y^2} +\frac{y^2z^2}{z^2(y+1)^2+y^2+y^2z^2} +\frac{z^2x^2}{x^2(z+1)^2+z^2+z^2x^2} \leq \frac{1}{2}. \]

\end{enumerate}


\vfill
% ASCII art
\centering
\small
\begin{BVerbatim}


    .--.              .--.
   : (\ ". _......_ ." /) :
    '.    `        `    .'
     /'   _        _   `\
    /     0}      {0     \
   |       /      \       |
   |     /'        `\     |
    \   | .  .==.  . |   /
     '._ \.' \__/ './ _.'
     /  ``'._-''-_.'``  \


\end{BVerbatim}

\end{document}

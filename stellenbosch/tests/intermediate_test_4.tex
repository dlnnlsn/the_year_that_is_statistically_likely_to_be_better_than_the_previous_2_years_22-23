\documentclass{article}

\usepackage{mathtools,amsfonts}
\usepackage{enumitem}
\usepackage{fullpage}
\usepackage{fancyvrb}
\usepackage{hyperref}


\begin{document}
\thispagestyle{empty}

\begin{center}
  \textbf{\Large Intermediate Test 4}
  % LEVEL is Senior, Intermediate or Beginner
  % NUMBER is the test number: 1, 2, etc.
  \\ \vspace{1em}
  \textbf{\large Stellenbosch Camp 2022}
  \\ \vspace{1em}
  \textbf{\large Time: $2\frac{1}{2}$ hours}
\end{center}

\bigskip

\begin{enumerate}[itemsep=\fill]

\item % Source of problem
Problem statement


\item %


\item % 


\item % 
Show that $2022 \times 2^n$ can be written as the sum of three distinct non-zero squares for every natural number $n$.

\item % Emile, 2022
You have an $n\times n$ chessboard which starts with all its squares white.
An operation involves choosing a row or column and flipping the colour of every square in this row or column between white and black.
Is it possible that for all $0 \leq k \leq n^2$, you can do some sequence of operations and end up with $k$ black squares?

\end{enumerate}


% ASCII art
\centering
\small
\begin{BVerbatim}
% Insert art here
\end{BVerbatim}

\end{document}
